In this chapter, we give a Gray code for generating Łukasiewicz words with fixed content using left-shifts. Section \ref{sec:lukasuc} will give the successor rule for the Gray code; Section \ref{sec:lukaproof} will prove its correctness.
\section{Successor Rule}\label{sec:lukasuc}

In this section, we provide a successor rule that applies a left-shift to a Łukasiewicz word.  The rule is given below in \eqref{eq:luka} 
Let $S$ be a multiset whose sum is equal to its length.  Let $\mathcal{L}(S)$ denote the set of valid Łukasiewicz words with content equal to S. Let $\alpha \in \mathcal{L}(S)$.  


Recall the definition of a left-shift from equation \eqref{eq:leftdef}: $\lshiftindex[\alpha]{i}{j}$ shifts the $j\thh$ symbol in $\alpha$ into the $i\thh$ position. 
% As in the cool-lex successor rule for multiset permutations in \ref{sec:coolPerms}, we let $x$ be the smallest value for which $\alpha_x<\alpha_{x-1}$.  
We also define $\rho=\alpha_1,\alpha_2,\cdots\alpha_{m}$ to be the non-increasing prefix of $\alpha$, i.e. the longest prefix of $\alpha$ such that $\alpha_{i}\ge\alpha_{i-1}$ for all $i \le m$.  
%In addition to the left-shift function, we define the length of the \emph{non-increasing prefix} of a string $\alpha$ to be 
%the maximum value $m$ such that $\alpha_{i-1} \ge \alpha_{i}$ for all $2 \le i \le m$. We will let $\rho$ $\rho = a_1 \cdot a_2 \cdots a_m$.  The sum of the symbols in $\rho$ is $\sum \rho = a_1 + a_2 + \cdots + a_m$


\begin{subnumcases}{\luka{\alpha} = \label{eq:luka}} 
    \lshiftindex{n}{2} & if  $m=n$ \label{eq:prefix_n_2}\\
	\lshiftindex{m+1}{1} & if  $m=n-1$  or  $\alpha_{m} < \alpha_{m+2}$   or \label{eq:prefix_m1_1}\\
	    & $(\alpha_{m+2} = 0 $ and $ \sum \rho = m) $\nonumber\\
	\lshiftindex{m+2}{1} &  if $ \alpha_{m+2} \neq 0 $\label{eq:prefix_m2_1}\\
	\lshiftindex{m+2}{2} & otherwise  \label{eq:prefix_m2_2}
\end{subnumcases}

% \begin{subnumcases}{\luka{\alpha} = \label{eq:luka}} 
%     \lshiftindex{n}{2} & if  $\alpha$ is in descending order \label{eq:prefix_n_2}\\
% 	\lshiftindex{x}{1} & if  $x=n$  or  $\alpha_{x-1} < \alpha_{x+w}$   or \label{eq:prefix_m1_1}\\
% 	    & $(\alpha_{x+1} = 0 $ and $ \sum \rho = x-1) $\nonumber\\
% 	\lshiftindex{m+2}{1} &  if $ \alpha_{m+2} \neq 0 $\label{eq:prefix_m2_1}\\
% 	\lshiftindex{m+2}{2} & otherwise  \label{eq:prefix_m2_2}
% \end{subnumcases}

Figure \ref{fig:LukaTable} illustrates the successor rule on every string in $\Lukas{S}$ for $S = \{0,0,0,1,2,3\}$.
For example, consider the top row with $\alpha = a_1 \cdot a_2 \cdot a_3 \cdot a_4 \cdot a_5 \cdot a_6 = 302100$.
Here the non-increasing prefix is $a_1 \cdot a_2 = 30$, so $x = 1$, and the length of the string is $n = 6$.
Thus, $x \neq n$, so \eqref{eq:prefix_n_2} is not applied.
Now consider the conditions in \eqref{eq:prefix_m1_1}.
The second condition is $a_{m} < a_{m+2}$, which is $a_2 = 0 < 1 = a_4$ for $\alpha$.  
Since this is true, $\luka{\alpha} = \lshiftindex{m+1}{1}$ by \eqref{eq:prefix_m1_1}, which is $\lshiftindex{3}{1}$ for $\alpha$.
In other words, the rule left-shifts $a_3$ into position $1$.
Thus, the next string in the list is $a_3 \cdot a_1 \cdot a_2 \cdot a_4 \cdot a_5 \cdot a_6 = 230100$, as seen in the second row of Figure~\ref{fig:LukaTable}.


\begin{figure}
    \centering
    % This figure is temporarily omitted because it is slow
    \begin{tabularx}{0.9\textwidth}{>{\hsize=0.275\hsize}C >{\hsize=0.225\hsize}C >{\hsize=0.05\hsize}C >{\hsize=0.1\hsize}C >{\hsize=0.2\hsize}C  >{\hsize=0.2\hsize}r }
\thead{Łukasiewicz path} & \thead{Łukasiewicz word} & \thead{$m$} & \thead{\eqref{eq:luka}} & \thead{shift} & \thead{scut} \\ \hline 
\LukaTable[2]{3,0,2,1,0,0} & 302100 & 2 & \eqref{eq:prefix_m1_1} & $\lshiftindex{3}{1}$ & $100$ \\
\LukaTable[3]{2,3,0,1,0,0} & 230100 & 1 & \eqref{eq:prefix_m2_2} & $\lshiftindex{3}{2}$ & $100$ \\
\LukaTable[2]{2,0,3,1,0,0} & 203100 & 2 & \eqref{eq:prefix_m1_1} & $\lshiftindex{3}{1}$ & $100$ \\
\LukaTable[3]{3,2,0,1,0,0} & 320100 & 3 & \eqref{eq:prefix_m2_2} & $\lshiftindex{5}{2}$ & $100$ \\
\LukaTable[2]{3,0,2,0,1,0} & 302010 & 2 & \eqref{eq:prefix_m2_2} & $\lshiftindex{4}{2}$ & $10$ \\
\LukaTable[2]{3,0,0,2,1,0} & 300210 & 3 & \eqref{eq:prefix_m1_1} & $\lshiftindex{4}{1}$ & $10$ \\
\LukaTable[3]{2,3,0,0,1,0} & 230010 & 1 & \eqref{eq:prefix_m2_2} & $\lshiftindex{3}{2}$ & $10$ \\
\LukaTable[2]{2,0,3,0,1,0} & 203010 & 2 & \eqref{eq:prefix_m1_1} & $\lshiftindex{3}{1}$ & $10$ \\
\LukaTable[3]{3,2,0,0,1,0} & 320010 & 4 & \eqref{eq:prefix_m2_2} & $\lshiftindex{6}{2}$ & $10$ \\
\LukaTable[2]{3,0,2,0,0,1} & 302001 & 2 & \eqref{eq:prefix_m2_2} & $\lshiftindex{4}{2}$ & $1$ \\
\LukaTable[2]{3,0,0,2,0,1} & 300201 & 3 & \eqref{eq:prefix_m1_1} & $\lshiftindex{4}{1}$ & $1$ \\
\LukaTable[3]{2,3,0,0,0,1} & 230001 & 1 & \eqref{eq:prefix_m2_2} & $\lshiftindex{3}{2}$ & $1$ \\
\LukaTable[2]{2,0,3,0,0,1} & 203001 & 2 & \eqref{eq:prefix_m1_1} & $\lshiftindex{3}{1}$ & $1$ \\
\LukaTable[3]{3,2,0,0,0,1} & 320001 & 5 & \eqref{eq:prefix_m1_1} & $\lshiftindex{6}{1}$ & $1$ \\
\LukaTable[3]{1,3,2,0,0,0} & 132000 & 1 & \eqref{eq:prefix_m1_1} & $\lshiftindex{2}{1}$ & $2000$ \\
\LukaTable[3]{3,1,2,0,0,0} & 312000 & 2 & \eqref{eq:prefix_m2_2} & $\lshiftindex{4}{2}$ & $2000$ \\
\LukaTable[2]{3,0,1,2,0,0} & 301200 & 2 & \eqref{eq:prefix_m1_1} & $\lshiftindex{3}{1}$ & $200$ \\
\LukaTable[2]{1,3,0,2,0,0} & 130200 & 1 & \eqref{eq:prefix_m1_1} & $\lshiftindex{2}{1}$ & $200$ \\
\LukaTable[2]{3,1,0,2,0,0} & 310200 & 3 & \eqref{eq:prefix_m2_2} & $\lshiftindex{5}{2}$ & $200$ \\
\LukaTable[2]{3,0,1,0,2,0} & 301020 & 2 & \eqref{eq:prefix_m2_2} & $\lshiftindex{4}{2}$ & $20$ \\
\LukaTable[2]{3,0,0,1,2,0} & 300120 & 3 & \eqref{eq:prefix_m1_1} & $\lshiftindex{4}{1}$ & $20$ \\
\LukaTable[2]{1,3,0,0,2,0} & 130020 & 1 & \eqref{eq:prefix_m1_1} & $\lshiftindex{2}{1}$ & $20$ \\
\LukaTable[2]{3,1,0,0,2,0} & 310020 & 4 & \eqref{eq:prefix_m1_1} & $\lshiftindex{5}{1}$ & $20$ \\
\LukaTable[3]{2,3,1,0,0,0} & 231000 & 1 & \eqref{eq:prefix_m2_1} & $\lshiftindex{3}{1}$ & $31000$ \\
\LukaTable[3]{1,2,3,0,0,0} & 123000 & 1 & \eqref{eq:prefix_m1_1} & $\lshiftindex{2}{1}$ & $3000$ \\
\LukaTable[3]{2,1,3,0,0,0} & 213000 & 2 & \eqref{eq:prefix_m2_2} & $\lshiftindex{4}{2}$ & $3000$ \\
\LukaTable[2]{2,0,1,3,0,0} & 201300 & 2 & \eqref{eq:prefix_m1_1} & $\lshiftindex{3}{1}$ & $300$ \\
\LukaTable[2]{1,2,0,3,0,0} & 120300 & 1 & \eqref{eq:prefix_m1_1} & $\lshiftindex{2}{1}$ & $300$ \\
\LukaTable[2]{2,1,0,3,0,0} & 210300 & 3 & \eqref{eq:prefix_m1_1} & $\lshiftindex{4}{1}$ & $300$ \\
\LukaTable[3]{3,2,1,0,0,0} & 321000 & 6 & \eqref{eq:prefix_n_2} & $\lshiftindex{6}{2}$ & $\epsilon$ 
    \end{tabularx}
    \caption{The left-shift Gray code $\cool{S}$ for Łukasiewicz words with content $S = \{0,0,0,1,2,3\}$.
    Each row gives the non-increasing prefix length $m$, the rule \eqref{eq:luka}, and the shift that creates the next word.
    The right column gives the scut of each string, which illustrates the suffix-based recursive definition of cool-lex order.}
    \label{fig:LukaTable}
\end{figure}

\subsection{Observations}
\label{sec:prefix_observations}

%We now make a couple of observations to give the reader a better sense of the rule. %how the rule works.
Note that \eqref{eq:luka} left-shifts a symbol that is at most two symbols past the non-increasing prefix. %, or the last symbol if there is no such symbol.
%This has two immediate consequences.
Thus, the shifts given by \eqref{eq:luka} are usually short, and the symbols at the right side of the string are rarely changed.
% TODO: should be a period here
%Second, the symbols at the right side of the string are rarely changed.
This implies that the order will have some similarity to co-lexicographic order, which orders strings right-to-left by increasing symbols. 
In fact, the order turns out to be a cool-lex order, as discussed in Section~\ref{sec:lukaproof}.

\section{Proof of Correctness}\label{sec:lukaproof}


This section proves that the successor rule in \eqref{eq:luka} generates all Łukasiewicz words with a given set of content.
% Now we prove that the successor rule is correct. %from Section \ref{sec:prefix} is correct.
%Given a Łukasiewicz word $\alpha \in \Lukas{S}$, the rule creates the next string a shift Gray code for the fixed-content set $\Lukas{S}$.
Our strategy is to define a recursive order of $\Lukas{S}$, and show that \eqref{eq:luka} creates the next string in this order.
%By doing so, we'll also see that our Gray code is a generalization of a previously published Gray code for Dyck words by Ruskey and Williams \cite{ruskey2008generating}.

\subsection{Terminology and Remarks}
\label{sec:proof_cool}

% \emph{Cool-lex order} is a variation of co-lexicographic order.
% The order was first given for \emph{$(s,t)$-combinations}, which are binary strings with $s$ copies of $0$ and $t$ copies of $1$, by Ruskey and Williams \cite{ruskey2005generating,ruskey2009coolest}.
% In this context, the order gives a \emph{prefix-shift Gray code}, meaning that a single symbol is left-shifted into the first position.
% The prefix-shift Gray code was then generalized to Dyck words \cite{ruskey2008generating} and multiset permutations \cite{CoolSODA}.
The cool-lex order for multiset permutations \cite{CoolSODA}, discussed in Section \ref{sec:coolPerms}, provides the recursive structure of our left-shift Gray code of fixed-content Łukasiewicz words. This section discusses the terminology used in describing the cool-lex order for multiset permutations.

\subsubsection{Tails and Scuts}
\label{sec:proof_scuts}

Given a multiset $S$ of cardinality $n$, we define the \emph{tail of length $\ell$} to be smallest $\ell$ symbols arranged in a string in non-increasing order.
Formally,
\begin{equation}
    \tail(\ell) = t_{\ell} \cdot t_{\ell-1} \cdots t_2 \cdot t_1,
\end{equation}
where $\tail(n) = t_n \cdot t_{n-1} \cdots t_1$ is the unique non-increasing string with content~$S$.

In English, a \emph{scut} is a short tail.
We use the term for a tail that is truncated by the addition of a large first symbol.
More specifically, a scut of length $\ell$ and a tail of length $\ell$ are identical, except for their first symbol, and the first symbol is larger in the scut. % (and also a member of $S$).
Formally, the \emph{scut of length $\ell+1$}, with respect to $S$ is
\begin{equation}
    \scut(s, \ell) = s \cdot \tail(\ell),
\end{equation}
where $s \in S$ is greater than the first symbol $\tail(\ell+1)$.
We refer to a scut of the form $\scut(s, \ell)$ as an \emph{$s$-scut}.

\subsubsection{Recursive Order}
\label{sec:proof_recursive}

Now we define $\cool{S}$ to be an order of $\Lukas{S}$.
More broadly, we define $\cool{S}$ on any multiset $S$ with non-negative symbols whose sum is at least as large as its cardinality, and we henceforth refer to these $S$ as \emph{valid}.
We define $\cool{S}$ recursively by grouping the strings with the same scut together.
Specifically, the scuts are ordered as follows:
\begin{itemize}
    \item The scuts are first ordered by their first symbol in increasing order.
    In other words, $s$-scuts are before $(s+1)$-scuts.
    % In other words, scuts starting with value $1$ appear before scuts starting with value $2$, and so on, up to the maximum value in $S$.
    %(There are no scuts starting with value $0$ since $0$ is never smaller than another symbol in $S$.)
    \item For a given first symbol, the scuts are ordered by decreasing length.
    In other words, longer $s$-scuts come before shorter $s$-scuts.
    \item The string $\tail(n)$ is the only string without a scut, and it is ordered last.
\end{itemize}
For example, the rightmost column of Figure \ref{fig:LukaTable} illustrates this order.
More specifically, the scuts appear in the following order:
\begin{equation} \label{eq:scuts}
100, 10, 1, 2000, 200, 20, 31000, 3000, 300,    
\end{equation}
% TODO: should these be parens instead of curly braces for tail{n} 
with the single string $\tail(n) = 321000$ appearing last.
Note that $2$, $30$ and $3$ are absent from \eqref{eq:scuts} because there are no Łukasiewicz words with these suffixes.

In each scut group the strings are ordered recursively.
In other words, the common scut is removed from the strings in a particular group, and then they are ordered according to $\cool{S'}$, where $S'$ is the valid multiset obtained by removing the symbols of the common scut from $S$.
For example, in Figure \ref{fig:LukaTable}, the strings with scut $1$ are ordered according to $\cool{S'}$ where $S' = \{3,2,1,0,0,0\} - \{1\} = \{3,2,0,0,0\}$.
The base case of the recursion is when $S = \emptyset$.

In the following subsection it will be helpful to know the first string that has an $s$-scut.
By our recursive order, we know that it will have a longest $s$-scut.
Moreover, the exact string can be obtained from the tail by a single shift.
To illustrate this, consider the list in Figure \ref{fig:LukaTable}, and let $\alpha = \tail(n) = 321000$.
\begin{itemize}
    \item The first string with a $1$-scut is $\lshiftindex[\alpha]{4}{2} = 302100$.
    \item The first string with a $2$-scut is $\lshiftindex[\alpha]{3}{1} = 132000$.
    \item The first string with a $3$-scut is $\lshiftindex[\alpha]{2}{1} = 231000$.
\end{itemize}
In other words, the first string with a $1$-scut is obtained by shifting a $0$ into the second position, while the first strings with $2$-scuts and $3$-scuts are obtained by shifting $1$ and $2$ into the first position, respectively.
This point is stated more generally in the following remark.

\begin{remark}
\label{rem:first}
Let $S$ be a valid multiset, and $\tail(n) = t_n \cdot t_{n-1} \cdots t_1$ with $t_i > t_{i-1}$.
The first string in $\cool{S}$ with a $t_i$-scut is $\lshiftindex[\tail(n)]{n-i+2}{1}$ if $t_{i-1} = 0$ or $\lshiftindex[\tail(n)]{n-i+2}{2}$ if $t_{i-1} > 0$.
\end{remark}

Less technically, Remark \ref{rem:first} says that the first string with a $t_i$-scut is obtained by shifting the next smallest symbol from its position in $\tail(n)$ into the first position, or second position if it is $0$.

% Now we formally define the order recursively. 
% In the definition, we use 
% \begin{subnumcases}{cool(S) =}
%     \phantom{LaTeX Sucks Center this Crap} \emptystring & if $S = \emptyset$ \\
%     \begin{array}{rcl}
%         cool(S - \scut(d_2,F_1-1)) & \cdot & \scut(d_2,F_1-1), \\
%         cool(S - \scut(d_2,F_1-2)) & \cdot & \scut(d_2,F_1-2), \\
%         & \vdots & \\
%         cool(S - \scut(d_2,0)) & \cdot & \scut(d_2,0), \\[0.5em]
%         cool(S - \scut(d_3,F_2-1)) & \cdot & \scut(d_3,F_2-1), \\
%         cool(S - \scut(d_3,F_2-2)) & \cdot & \scut(d_3,F_2-2), \\
%         & \vdots & \\
%         cool(S - \scut(d_3,0)) & \cdot & \scut(d_3,0), \\
%         & {\Large \vdots} & \\
%         cool(S - \scut(d_k,F_{k-1}-1)) & \cdot & \scut(d_k,F_{k-1}-1), \\
%         cool(S - \scut(d_k,F_{k-1}-2)) & \cdot & \scut(d_k,F_{k-1}-2), \\
%         & \vdots & \\
%         cool(S - \scut(d_k,0)) & \cdot & \scut(d_k,0), \\
%         & & \tail(n) \\
%     \end{array} & 
%     \begin{tabular}{c}
%         otherwise, 
%     \end{tabular}
% \end{subnumcases}
% where $sort(S) = d_1, d_2, \ldots, d_k$ and $f_i$ for frequencies and $F_i$ for cumulative frequencies

% The following remark is due to the fact that the above ordering of the scuts (and $tail{n}$) include every possible suffix of a string in $\Lukas{S}$

% \begin{remark}
% \label{rem:ordering}
% $\cool{S}$ is an ordering of $\Lukas{S}$.
% \end{remark}

\subsection{Equivalence}
\label{sec:proof_equal}

Now we prove that the successor rule \eqref{eq:luka} correctly provides the next string in $\cool{S}$.
This simultaneously proves that \eqref{eq:luka} is a successor rule for a left-shift Gray code of $\Lukas{S}$, and that $\cool{S}$ is a recursive description of the same.

\begin{theorem}
\label{thm:equal}
Let $S$ be a multiset of non-negative values with cardinality $n$ and sum $\Sigma S = n$.
Also, let $\alpha \in \Lukas{S}$ be a Łukasiewicz word with content $S$, and $\beta \in \Lukas{S}$ be the next string in $\cool{S}$ taken circularly (i.e., if $\alpha$ is the last string in $\cool{S}$, then $\beta$ is the first string in $\cool{S}$).
Then $\beta = \lshiftindex[\alpha]{j}{i}$.
In other words, the successor rule in \eqref{eq:luka} transforms $\alpha$ into $\beta$ with a left-shift.
\end{theorem}
\begin{proof}
Let $\alpha = a_1 \cdot a_2 \cdots a_n$ and $\rho = a_1 \cdot a_2 \cdots a_m$ be $\alpha$'s non-increasing prefix.

\begin{itemize}[nosep]
\item If $m=n$, then $\alpha = \tail(n)$ and it is the last string in $\cool{S}$.
We also know that $\nextPrefix{\alpha} = \lshiftindex{n}{2}$ by \eqref{eq:prefix_n_2}.
This gives the first string in $\cool{S}$ with a $1$-scut by Remark \ref{rem:first}, which is the first string in $\cool{S}$ as expected.
This is the only case where \eqref{eq:prefix_n_2} is used.
\item If $m=n-1$, then $\alpha$'s non-increasing prefix extends until its second-last symbol.
Furthermore, we know that $a_n = 1$, since this is the only non-zero value that can appear in the rightmost position.
We also know that $\nextPrefix{\alpha} = \lshiftindex{m+1}{1} = \lshiftindex{n}{1}$ by \eqref{eq:prefix_m1_1}.
Thus, Remark \ref{rem:first} implies that $\beta$ is the first string with an $x$-scut, where $x$ is the smallest symbol larger than $1$ in $S$.
This is expected since $\alpha$ is the last string in the order with a $1$-scut.
\end{itemize}
\noindent
The remaining cases are handled cumulatively (i.e., each assumes that the previous do not hold).
Note that $\alpha = \rho \cdot a_{m+1} \cdot a_{m+2} \cdots a_n$ is the last string with $\scut(a_{m+1}, \ell) = a_{m+1} \cdot a_{m+2} \cdots a_{w}$ in a sublist $\cool{S- \{a_{w+1}, a_{w+2}, \ldots, a_n\}}$.
We also view $\lshiftindex[\alpha]{j}{i}$ in two steps:
$a_j$ is left-shifted until it joins the non-increasing prefix, then further to index $i$.
This allows us to use Remark \ref{rem:first}. 
\begin{itemize}[nosep]
    \item If $a_m < a_{m+2}$, then the scut at this level of recursion, namely $\scut(a_{m+1}, \ell)$, cannot be shortened since $\ell=0$.
    So the next scut will be the longest scut with the next largest symbol, which is true by Remark \ref{rem:first} and $\nextPrefix{\alpha} = \lshiftindex{m+1}{1}$ by \eqref{eq:prefix_m1_1}.
    \item If $a_{m+2} = 0$ and $\Sigma \rho = m$, then the scut cannot be shortened since the sum of the symbols before the shorter scut will be less than their cardinality.
    Thus, the next scut will be the longest scut with the next largest symbol, which is true by Remark \ref{rem:first} and $\nextPrefix{\alpha} = \lshiftindex{m+1}{1}$ from  \eqref{eq:prefix_m1_1}.
    \item If $a_{m+2} \neq 0$, then the scut at this level of recursion can be shortened to $\scut(a_{m+1}, \ell-1)$.
    Given this shorter scut, the order recursively adds new scuts beginning with the first $x$-scut, where $x$ is the second-smallest remaining symbol.
    This is true by Remark \ref{rem:first} and $\nextPrefix{\alpha} = \lshiftindex{m+2}{1}$ by \eqref{eq:prefix_m2_1}.
    \item Otherwise, $a_{m+2} = 0$.
    This is identical to the previous case, except that $a_{m+2} = 0$.
    Thus, Remark \ref{rem:first} gives $\nextPrefix{\alpha} = \lshiftindex{m+2}{2}$ by \eqref{eq:prefix_m2_2}
\end{itemize}
Therefore, \eqref{eq:luka} gives the next string in the order, which completes the proof.
\end{proof}

