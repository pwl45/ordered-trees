The algorithm described in this section has two primary implementations: The first implementation uses a linked structure for the nodes, where an ordered tree node stores pointers to its leftmost child and its right sibling.  Therefore, to access a node's 3rd child, one would use 

\verb+node->left_child->right_sibling->right_sibling+ 

This has the advantages of space efficiency and $O(1)$ appending and prepending to a node's list of children, but the disadvantage of requiring $O(k)$ time to access a node's $k\thh$ child. 

The second implementation uses an array-like approach to storing children.  This has the advantage of $O(1)$ access time for all children but the disadvantage of either requiring $O(n)$ space for each node or the additional cost of resizing arrays.  Each node also stores a counter for its number of children and an int keeping track of the maximum amount of children it can store (before requiring reallocation)).  This implementation stores the list of children ``backwards,'' so to access a node's $k\thh$ child, one would use \verb+node->children[(node->nch)-k]+.  Each \verb+node->children[0]+ is set to and kept as \verb+NULL+ for a null-termination like effect that is useful in the algorithm.
% TODO NEW: initial tree
\begin{figure}
    \begin{subfigure}[t]{.5 \textwidth}
	\begin{center}

	    \begin{Verbatim}
typedef struct node {
    int data;
    struct node* parent;
    struct node* left_child;
    struct node* right_sibling;
} node;

node* new_node(int data){
    node* n = 
	(node*)malloc(sizeof(node));
    n->data=data;
    n->parent=NULL;
    n->left_child=NULL;
    n->right_sibling=NULL;
    return n;
}
	    \end{Verbatim}
	\end{center}

	\label{fig:}
    \end{subfigure}
    \begin{subfigure}[t]{.5 \textwidth}
	\begin{center}

	    \begin{Verbatim}
typedef struct arraynode {
    int nch; 
    int maxch;
    struct arraynode* parent;
    struct arraynode** children;
} anode;

anode* new_anode(int maxch){
    anode* n = 
	(anode*)malloc(sizeof(anode));
    n->maxch=maxch;
    n->parent=NULL;
    n->nch=0;
    n->children =
	(anode**)malloc(sizeof(anode*) * maxch+1);
    n->children[0]=NULL; 
    return n;
}
	    \end{Verbatim}
	\end{center}
	% \caption{ $C$ implementation of coolOtree} 
	\label{fig:}
    \end{subfigure}

    % TODO: change bit colors
    \cprotect\caption{C code for two different ordered tree structures and functions to initialize a node.}
    \label{fig:otreeCode}
\end{figure}

% TODO: QUESTION: Is this confusing? Using t like this simplifies the correspondence between ordered trees and Dyck words. But would it be clearer to have t just be the number of total nodes?

% Verbatim (not verbatim) from fancyvrb package lets you do \ref inside of it. 

\begin{figure}
    \centering
\begin{tikzpicture}[every tree node/.style={draw,circle},sibling distance=10pt, level distance=40pt]
\tikzset{edge from parent/.style={draw, edge from parent path=
    {(\tikzparentnode) -- (\tikzchildnode)}}}
    \Tree [.{} [.{} ] ] 
\end{tikzpicture} \hspace{1.5em}
\begin{tikzpicture}[every tree node/.style={draw,circle},sibling distance=10pt, level distance=40pt]
\tikzset{edge from parent/.style={draw, edge from parent path=
    {(\tikzparentnode) -- (\tikzchildnode)}}}
    \Tree [.{} [.{} ] [.{} ] ] 
\end{tikzpicture}\hspace{1.5em}
\begin{tikzpicture}[every tree node/.style={draw,circle},sibling distance=10pt, level distance=40pt]
\tikzset{edge from parent/.style={draw, edge from parent path=
    {(\tikzparentnode) -- (\tikzchildnode)}}}
    \Tree [.{} [.{} ] [.{} [.{} ] ] ] 
\end{tikzpicture}\hspace{1.5em}
\begin{tikzpicture}[every tree node/.style={draw,circle},sibling distance=10pt, level distance=40pt]
\tikzset{edge from parent/.style={draw, edge from parent path=
    {(\tikzparentnode) -- (\tikzchildnode)}}}
    \Tree [.{} [.{} ] [.{} [.{} [.{} ] ] ] ] 
\end{tikzpicture}

    \cprotect\caption{The initial trees returned by \verb$get_initial_tree$ in \verb$coolOtree$ with $t=1,2,3,$ and $4$ \\
    Note the convenient feature that for each tree, O is equal to \verb$root->left_child->right_sibling$
    }
    \label{exampleotree}
\end{figure}



% TODO NEW: initial tree
\begin{figure}
    \centering
    \begin{subfigure}[t]{.49 \textwidth}
	\begin{center}
	    \begin{algorithm}[H] % What does the H do? it makes it work.
	    \begin{algorithmic}
    \Function{cool-ordered-trees}{$t$}
        % \vspace{0.75em}
        \LineComment{Generate initial tree} % TODO
	\State $O\gets root.lchild$
	\State \visit{root}
        \While{$O \ne NULL$}
	    \State $P \gets O.parent$
	    \If{$O.lchild \ne NULL$}
		\State $\pushchild{O}{\popchild{P}}$
		\State $O \gets O.lchild.rsibling$
	    \Else
		\If{$O.parent == root$}
		\State $\pushchild{O}{\popchild{P}}$
		\Else
		\State $\pushchild{O}{\popchild{P}}$
		\State $\pushchild{root}{\popchild{P}}$
		\EndIf
		\State $O \gets O.rsibling$
	    \EndIf
	\State \visit{root}
        \EndWhile
    \EndFunction
	    \end{algorithmic}
    \caption*{Generate all ordered trees with $t+1$ nodes}
	\end{algorithm}
	\end{center}
	% \caption{Pseudoc}
	\label{fig:}
    \end{subfigure}
    \begin{subfigure}[t]{.5 \textwidth}
	\begin{center}
	    \vspace{.9em} % fun alignment
	    % \vspace{2.1em} % fun alignment
\begin{Verbatim}[commandchars=\\\[\]]

void coolOtree(int t, void (*visit)(node*)){
    node* root = get_initial_tree(t);
    node* o=root->left_child->right_sibling;
    visit(root);
    while(o){
	node* p = o->parent;
	if(o->left_child){ 
	    pushchild(o,popchild(p));
	    o=o->left_child->right_sibling;
	}else{
	    if(o->parent == root){ 
		pushchild(o,popchild(p));
	    }else{ 
		pushchild(p->parent,popchild(p));
		pushchild(root,popchild(p));
	    }
	    o=o->right_sibling;
	}
	visit(root);
    }
}
\end{Verbatim}
	\end{center}
	% \caption{ $C$ implementation of coolOtree} 
	\label{fig:}
    \end{subfigure}

    % TODO: change bit colors
    \cprotect\caption{Pseudocode and C implementation}
    \label{fig:otreeCode}
\end{figure}
