% reqno is needed to make the subnumcases have the case numbers on the right
% \documentclass[reqno]{amsart}
\documentclass[twoside,reqno]{report}
% \usepackage[top=1in, bottom=1in, left=1.5in, right=1in, includehead]{geometry}
% \documentclass{article}
% \documentclass[runningheads]{llncs}
% \documentclass[preview]{standalone}
\usepackage{twoopt}
% \usepackage{bookmark}
\usepackage{tabularx,booktabs}
% \usepackage{array}
\usepackage{enumitem} % how does this affect itemize? shouldn't this be its own thing instead of an option for itemize????
\renewcommand\tabularxcolumn[1]{m{#1}}
\newcommand{\LukaTable}[2][2]{\LukaTableDriver[0.36][0.18]{6}{#1}{#2}}
\newcommand{\DyckTable}[2][2]{\LukaTableDriver[.36][.22]{8}{#1}{#2}}


\newcommand{\C}{\mathcal{C}}
\newcommand{\D}{\mathbf{D}}
\newcommand{\luka}[1]{\overleftarrow{\text{luka}}(#1)}

\newcommand{\octoSliceTable}[1]{

    \begin{tikzpicture}
        % Define the heptagon coordinates
        \coordinate (A) at(0.4362591129,1.0532226671);
        \coordinate (B) at(-0.4362591129,1.0532226671);
        \coordinate (C) at(-1.0532226671,0.4362591129);
        \coordinate (D) at(-1.0532226671,-0.4362591129);
        \coordinate (E) at(-0.4362591129,-1.0532226671);
        \coordinate (F) at(0.4362591129,-1.0532226671);
        \coordinate (G) at(1.0532226671,-0.4362591129);
        \coordinate (H) at(1.0532226671,0.4362591129);

        \coordinate (1) at(0.4362591129,1.0532226671);
        \coordinate (2) at(-0.4362591129,1.0532226671);
        \coordinate (3) at(-1.0532226671,0.4362591129);
        \coordinate (4) at(-1.0532226671,-0.4362591129);
        \coordinate (5) at(-0.4362591129,-1.0532226671);
        \coordinate (6) at(0.4362591129,-1.0532226671);
        \coordinate (7) at(1.0532226671,-0.4362591129);
        \coordinate (8) at(1.0532226671,0.4362591129);

        \octoslice{{#1}}
        \draw[densely dotted] (E) -- (F);

    \end{tikzpicture}
}
\newcommand{\octoSliceTableA}{

    \begin{tikzpicture}
        % Define the heptagon coordinates
        \coordinate (A) at(0.4362591129,1.0532226671);
        \coordinate (B) at(-0.4362591129,1.0532226671);
        \coordinate (C) at(-1.0532226671,0.4362591129);
        \coordinate (D) at(-1.0532226671,-0.4362591129);
        \coordinate (E) at(-0.4362591129,-1.0532226671);
        \coordinate (F) at(0.4362591129,-1.0532226671);
        \coordinate (G) at(1.0532226671,-0.4362591129);
        \coordinate (H) at(1.0532226671,0.4362591129);

        \coordinate (1) at(0.4362591129,1.0532226671);
        \coordinate (2) at(-0.4362591129,1.0532226671);
        \coordinate (3) at(-1.0532226671,0.4362591129);
        \coordinate (4) at(-1.0532226671,-0.4362591129);
        \coordinate (5) at(-0.4362591129,-1.0532226671);
        \coordinate (6) at(0.4362591129,-1.0532226671);
        \coordinate (7) at(1.0532226671,-0.4362591129);
        \coordinate (8) at(1.0532226671,0.4362591129);

        \octosliceA{}
        \draw[densely dotted] (E) -- (F);

    \end{tikzpicture}
}

\newcommand{\octoSliceTableB}{

    \begin{tikzpicture}
        % Define the heptagon coordinates
        \coordinate (A) at(0.4362591129,1.0532226671);
        \coordinate (B) at(-0.4362591129,1.0532226671);
        \coordinate (C) at(-1.0532226671,0.4362591129);
        \coordinate (D) at(-1.0532226671,-0.4362591129);
        \coordinate (E) at(-0.4362591129,-1.0532226671);
        \coordinate (F) at(0.4362591129,-1.0532226671);
        \coordinate (G) at(1.0532226671,-0.4362591129);
        \coordinate (H) at(1.0532226671,0.4362591129);

        \coordinate (1) at(0.4362591129,1.0532226671);
        \coordinate (2) at(-0.4362591129,1.0532226671);
        \coordinate (3) at(-1.0532226671,0.4362591129);
        \coordinate (4) at(-1.0532226671,-0.4362591129);
        \coordinate (5) at(-0.4362591129,-1.0532226671);
        \coordinate (6) at(0.4362591129,-1.0532226671);
        \coordinate (7) at(1.0532226671,-0.4362591129);
        \coordinate (8) at(1.0532226671,0.4362591129);

        \octosliceB{}
        \draw[densely dotted] (E) -- (F);

    \end{tikzpicture}
}

\newcommand{\octoSliceTableC}{

    \begin{tikzpicture}
        % Define the heptagon coordinates
        \coordinate (A) at(0.4362591129,1.0532226671);
        \coordinate (B) at(-0.4362591129,1.0532226671);
        \coordinate (C) at(-1.0532226671,0.4362591129);
        \coordinate (D) at(-1.0532226671,-0.4362591129);
        \coordinate (E) at(-0.4362591129,-1.0532226671);
        \coordinate (F) at(0.4362591129,-1.0532226671);
        \coordinate (G) at(1.0532226671,-0.4362591129);
        \coordinate (H) at(1.0532226671,0.4362591129);

        \coordinate (1) at(0.4362591129,1.0532226671);
        \coordinate (2) at(-0.4362591129,1.0532226671);
        \coordinate (3) at(-1.0532226671,0.4362591129);
        \coordinate (4) at(-1.0532226671,-0.4362591129);
        \coordinate (5) at(-0.4362591129,-1.0532226671);
        \coordinate (6) at(0.4362591129,-1.0532226671);
        \coordinate (7) at(1.0532226671,-0.4362591129);
        \coordinate (8) at(1.0532226671,0.4362591129);

        \octosliceC{}
        \draw[densely dotted] (E) -- (F);

    \end{tikzpicture}
}

\newcommand{\octoSliceTableD}{

    \begin{tikzpicture}
        % Define the heptagon coordinates
        \coordinate (A) at(0.4362591129,1.0532226671);
        \coordinate (B) at(-0.4362591129,1.0532226671);
        \coordinate (C) at(-1.0532226671,0.4362591129);
        \coordinate (D) at(-1.0532226671,-0.4362591129);
        \coordinate (E) at(-0.4362591129,-1.0532226671);
        \coordinate (F) at(0.4362591129,-1.0532226671);
        \coordinate (G) at(1.0532226671,-0.4362591129);
        \coordinate (H) at(1.0532226671,0.4362591129);

        \coordinate (1) at(0.4362591129,1.0532226671);
        \coordinate (2) at(-0.4362591129,1.0532226671);
        \coordinate (3) at(-1.0532226671,0.4362591129);
        \coordinate (4) at(-1.0532226671,-0.4362591129);
        \coordinate (5) at(-0.4362591129,-1.0532226671);
        \coordinate (6) at(0.4362591129,-1.0532226671);
        \coordinate (7) at(1.0532226671,-0.4362591129);
        \coordinate (8) at(1.0532226671,0.4362591129);

        \octosliceD{}
        \draw[densely dotted] (E) -- (F);

    \end{tikzpicture}
}
\newcommand{\octoSliceTableE}{

    \begin{tikzpicture}
        % Define the heptagon coordinates
        \coordinate (A) at(0.4362591129,1.0532226671);
        \coordinate (B) at(-0.4362591129,1.0532226671);
        \coordinate (C) at(-1.0532226671,0.4362591129);
        \coordinate (D) at(-1.0532226671,-0.4362591129);
        \coordinate (E) at(-0.4362591129,-1.0532226671);
        \coordinate (F) at(0.4362591129,-1.0532226671);
        \coordinate (G) at(1.0532226671,-0.4362591129);
        \coordinate (H) at(1.0532226671,0.4362591129);

        \coordinate (1) at(0.4362591129,1.0532226671);
        \coordinate (2) at(-0.4362591129,1.0532226671);
        \coordinate (3) at(-1.0532226671,0.4362591129);
        \coordinate (4) at(-1.0532226671,-0.4362591129);
        \coordinate (5) at(-0.4362591129,-1.0532226671);
        \coordinate (6) at(0.4362591129,-1.0532226671);
        \coordinate (7) at(1.0532226671,-0.4362591129);
        \coordinate (8) at(1.0532226671,0.4362591129);

        \octosliceE{}
        \draw[densely dotted] (E) -- (F);

    \end{tikzpicture}
}
\newcommand{\octoSliceTableF}{

    \begin{tikzpicture}
        % Define the heptagon coordinates
        \coordinate (A) at(0.4362591129,1.0532226671);
        \coordinate (B) at(-0.4362591129,1.0532226671);
        \coordinate (C) at(-1.0532226671,0.4362591129);
        \coordinate (D) at(-1.0532226671,-0.4362591129);
        \coordinate (E) at(-0.4362591129,-1.0532226671);
        \coordinate (F) at(0.4362591129,-1.0532226671);
        \coordinate (G) at(1.0532226671,-0.4362591129);
        \coordinate (H) at(1.0532226671,0.4362591129);

        \coordinate (1) at(0.4362591129,1.0532226671);
        \coordinate (2) at(-0.4362591129,1.0532226671);
        \coordinate (3) at(-1.0532226671,0.4362591129);
        \coordinate (4) at(-1.0532226671,-0.4362591129);
        \coordinate (5) at(-0.4362591129,-1.0532226671);
        \coordinate (6) at(0.4362591129,-1.0532226671);
        \coordinate (7) at(1.0532226671,-0.4362591129);
        \coordinate (8) at(1.0532226671,0.4362591129);

        \octosliceF{}
        \draw[densely dotted] (E) -- (F);

    \end{tikzpicture}
}
\newcommand{\octoSliceTableG}{

    \begin{tikzpicture}
        % Define the heptagon coordinates
        \coordinate (A) at(0.4362591129,1.0532226671);
        \coordinate (B) at(-0.4362591129,1.0532226671);
        \coordinate (C) at(-1.0532226671,0.4362591129);
        \coordinate (D) at(-1.0532226671,-0.4362591129);
        \coordinate (E) at(-0.4362591129,-1.0532226671);
        \coordinate (F) at(0.4362591129,-1.0532226671);
        \coordinate (G) at(1.0532226671,-0.4362591129);
        \coordinate (H) at(1.0532226671,0.4362591129);

        \coordinate (1) at(0.4362591129,1.0532226671);
        \coordinate (2) at(-0.4362591129,1.0532226671);
        \coordinate (3) at(-1.0532226671,0.4362591129);
        \coordinate (4) at(-1.0532226671,-0.4362591129);
        \coordinate (5) at(-0.4362591129,-1.0532226671);
        \coordinate (6) at(0.4362591129,-1.0532226671);
        \coordinate (7) at(1.0532226671,-0.4362591129);
        \coordinate (8) at(1.0532226671,0.4362591129);

        \octosliceG{}
        \draw[densely dotted] (E) -- (F);

    \end{tikzpicture}
}
\newcommand{\octoSliceTableH}{

    \begin{tikzpicture}
        % Define the heptagon coordinates
        \coordinate (A) at(0.4362591129,1.0532226671);
        \coordinate (B) at(-0.4362591129,1.0532226671);
        \coordinate (C) at(-1.0532226671,0.4362591129);
        \coordinate (D) at(-1.0532226671,-0.4362591129);
        \coordinate (E) at(-0.4362591129,-1.0532226671);
        \coordinate (F) at(0.4362591129,-1.0532226671);
        \coordinate (G) at(1.0532226671,-0.4362591129);
        \coordinate (H) at(1.0532226671,0.4362591129);

        \coordinate (1) at(0.4362591129,1.0532226671);
        \coordinate (2) at(-0.4362591129,1.0532226671);
        \coordinate (3) at(-1.0532226671,0.4362591129);
        \coordinate (4) at(-1.0532226671,-0.4362591129);
        \coordinate (5) at(-0.4362591129,-1.0532226671);
        \coordinate (6) at(0.4362591129,-1.0532226671);
        \coordinate (7) at(1.0532226671,-0.4362591129);
        \coordinate (8) at(1.0532226671,0.4362591129);

        \octosliceH{}
        \draw[densely dotted] (E) -- (F);

    \end{tikzpicture}
}
\newcommandtwoopt{\LukaTableDriver}[5][1][1]{
  \begin{tikzpicture}[x=#1cm, y=#2cm, step=1]
      \tikzmath{
        \xValue = 0;
        \yValue = 0;
        {
          \draw[black,thin] (0,0) grid (#3,#4);
          \filldraw[fill=none, draw=none] (#3, #4) ellipse (0.05cm and 0.05cm);  % for consistent bounding box
        };
        for \yDelta in {#5}{
          %\labelUp = int(\yDelta+1);
          %\labelDn = int(\yDelta);
          {
            \filldraw[fill=black, draw=black] (\xValue, \yValue) ellipse (0.05cm and 0.05cm);
            %\node[] at (\xValue + 0.5, -1) {\labelDn};
            %\node[] at (\xValue + 0.5, #3) {\labelUp};
          };
          \xNext = \xValue + 1;
          \yNext = \yValue + \yDelta - 1; % Here is the -1
          {
            \draw[black,very thick] (\xValue, \yValue) -- (\xNext, \yNext);
          };
          \xValue = \xNext;
          \yValue = \yNext;
        };
      }
      {
        \filldraw[fill=black, draw=black] (\xValue, \yValue) circle (0.05cm);
      };
  \end{tikzpicture}
}

% \newcommand*{\thead}[1]{%
% \multicolumn{1}{c}{\bfseries\begin{tabular}{@{}c@{}}#1\end{tabular}}}
\newcolumntype{C}{>{\centering\arraybackslash}X}

\usepackage{mathtools}
\usepackage[english]{babel}

% Order matters here, god knows why
\usepackage[dvipsnames]{xcolor}
\usepackage{tikz}

% stuff for making verbatim work better
\usepackage{fancyvrb}
\usepackage{cprotect}

\definecolor{p1col1}{rgb}{1.0, 0, 0}
% \definecolor{p1col1}{RGB}{80, 0, 130}
\definecolor{p1col2}{rgb}{0, 1.0, 0}
% \definecolor{p1col2}{RGB}{255, 190, 10}
\definecolor{p1col3}{rgb}{0, 0, 1.0}
% \definecolor{p1col3}{RGB}{255, 190, 10}
\definecolor{p1col4}{rgb}{0, 0, 0}
\definecolor{lightpurple}{RGB}{203, 195, 227}
% \definecolor{midpurple}{RGB}{243, 225, 247}
\definecolor{lightblue}{RGB}{173, 216, 230}
\definecolor{lightyellow}{RGB}{255, 255, 158}
\definecolor{lightred}{RGB}{255, 74, 71}
\definecolor{fgreen}{RGB}{0,69,33}
\definecolor{lightgreen}{RGB}{42,197,152}
\definecolor{midishgreen}{RGB}{19,170,121}
\definecolor{darkpurple}{RGB}{90,25,214}
\definecolor{midpurp}{RGB}{123,66,233}
\definecolor{lmidpurp}{RGB}{170,133,241}
% \def{catleft}{BrickRed}
\definecolor{catleftdark}{RGB}{66,155,205}
\definecolor{catleft}{RGB}{86,175,205}
% \definecolor{catleft}{RGB}{106,195,205}
\definecolor{catmid}{RGB}{42,43,45}
\definecolor{catright}{RGB}{255,111,97}

% \definecolor{p1col4}{RGB}{40, 0, 80}

\usepackage{graphicx}

\usepackage{verbatim}
\usepackage{amsfonts}
\usepackage{amssymb}
\usepackage{amsmath}
\usepackage{amsthm}

\usepackage{colortbl}

\usepackage{algorithm}
\usepackage[noend]{algpseudocode}


\usepackage{twoopt}
\usepackage{tikz}
\usepackage{tikz-qtree}
\usetikzlibrary{math,shapes.geometric}

% \usepackage{caption}
\usepackage[justification=centering]{caption}
\usepackage{subcaption}

\usepackage[title]{appendix}

\usepackage{verbatim}

\newcommand{\emptystring}{\varepsilon}

\usepackage{cases}





%
% Algorithm environment commands
%
% https://tex.stackexchange.com/questions/74880/algorithmicx-package-comments-on-a-single-line
\algnewcommand{\LineComment}[1]{\State \(\triangleright\) #1}

% https://tex.stackexchange.com/questions/184154/algorithmic-put-if-and-endif-into-same-line
\algnewcommand{\OneLineIf}[2]{\State\algorithmicif\ #1\ \algorithmicthen\ #2}

% https://tex.stackexchange.com/questions/234690/and-in-algorithm
\algnewcommand{\algorithmicand}{\textbf{ and }}
\algnewcommand{\algorithmicor}{\textbf{ or }}
\algnewcommand{\algorithmicnot}{\textbf{ not }}
\algnewcommand{\OR}{\algorithmicor}
\algnewcommand{\AND}{\algorithmicand}
\algnewcommand{\NOT}{\algorithmicnot}

\newcommand{\visit}[1]{\texttt{visit}(#1)}
% \newcommand{\visit}[1]{\texttt{visit}(#1)}

% \newcommand{\pshift}[2][]{$\mathsf{preshift}_{#1}(#2)$}
\newcommand{\transpose}[3]{\mathsf{transpose(} #1, #2, #3 \mathsf{)}}
\newcommand{\preshift}[2]{\mathsf{preshift}_{#1}(#2 \mathsf{)}}

% \newcommand{\leftshift}[2]{\mathsf{leftshift(} #1, #2 \mathsf{)}}
\newcommand{\treeshift}[3]{\mathsf{shiftfirst(} #1, #2, #3 \mathsf{)}}
\newcommand{\popchild}[1]{\mathsf{popchild(} #1 \mathsf{)}}
\newcommand{\pushchild}[2]{\mathsf{pushchild(} #1, #2 \mathsf{)}}
% \newcommand{\poppush}[2]{\mathsf{poppush(} #1, #2 \mathsf{)}}
\newcommand{\poppush}[3][]{\mathsf{poppush}_{#1}(#2,#3)}
% \newcommand{\poppushopt}[2][]{\mathsf{poppush}_{#1}( #2, #3 \mathsf{)}}


% https://tex.stackexchange.com/questions/228474/bold-horizontally-and-vertically-aligned-multiline-table-headers
\newcommand*{\thead}[1]{%
\multicolumn{1}{c}{\bfseries\begin{tabular}{@{}c@{}}#1\end{tabular}}}



\newcommand{\nextPrefix}[1]{\mathsf{next}{(#1)}}
\newcommand{\nextTree}[1]{\mathsf{nextree}{(#1)}}
% \newcommand{\nextTree0}[0]{\mathsf{nextree}{}
\newcommand{\coolCat}[1]{\overleftarrow{\mathsf{coolCat}}{(#1)}}
% \newcommand{\coolCat0}[0]{\overrightarrow{\mathsf{coolCat}}{(#1)}}
\newcommand{\otree}[1]{\mathsf{OTree}{(#1)}}
\newcommand{\dyck}[1]{\mathsf{Dyck}{(#1)}}
\newcommand{\dyckindex}[1]{\mathsf{DyckIndex}{(#1)}}

\newcommand{\otreenode}[1]{\mathsf{OTreeNode}{(#1)}}
\newcommand{\leftdown}[1]{\mathsf{leftpath}{(#1)}}
\newcommand{\depth}[1]{\mathsf{Depth}{(#1)}}
% \newcommand{\ithone}[2]{\mathsf{i^{\underline{th}}one}_{#1}{(#2)}}
\newcommand{\oneindex}[2]{\mathsf{oneindex}({#1}, #2)}
\newcommand{\thh}{^{\underline{th}}}
% \newcommand{\path}[2]{\mathsf{path}({#1}, #2)}
\newcommand{\path}[3]{\mathsf{path({#1}, #2, #3)}}



\usepackage{tabularx,booktabs}
\usepackage[letterpaper,top=2cm,bottom=2cm,left=3cm,right=3cm,marginparwidth=1.75cm]{geometry}

\usepackage{amsmath}
\usepackage{graphicx}
\usepackage[colorlinks=true, allcolors=blue]{hyperref}

% THEOREM STUFF
\newtheorem{theorem}{Theorem}[chapter]
\newtheorem{lemma}[theorem]{Lemma}
% \newtheorem{theorem}{Theorem}[section]
\newtheorem{remark}{Remark}[chapter]
% \newtheorem{lemma}[theorem]{Lemma}

% polygon stuff: thank you to https://texample.net/tikz/examples/polygon-division/
% Macro for drawing a heptagon   
\def\hepta{\draw(A) -- (B) -- (C) -- (D) -- (E) -- (F) -- (G) -- cycle;}
\def\octo{\draw(A) -- (B) -- (C) -- (D) -- (E) -- (F) -- (G) -- (H) -- (A);}
\def\penta{\draw(a) -- (b) -- (c) -- (d) -- (e) -- cycle;}

% Macro for drawing polygon diagonals. 
% Example \slice{A/C,C/E,E/G,C/G}

\newcommand{\slice}[1]{%
    \hepta
    \draw \foreach \x/\y in {#1} {(\x)--(\y)};

}
\newcommand{\octosliceA}{%
  \draw[draw,fill=catleft](A) -- (B) -- (C) -- (D) -- (E) -- (A);
  \draw[draw, fill=catright](F) -- (G) -- (H) -- (A);
  \draw[draw,fill=catmid] (A) -- (E) -- (F) -- (A);
}

\newcommand{\octosliceB}{%
  \draw[draw,fill=catleft](B) -- (C) -- (D) -- (E) -- (B);
  \draw[draw, fill=catright](F) -- (G) -- (H) -- (A) -- (B);
  \draw[draw,fill=catmid] (B) -- (E) -- (F) -- (B);
}
\newcommand{\octosliceC}{%
  \draw[draw,fill=catleft](C) -- (D) -- (E) -- (C);
  \draw[draw, fill=catright](F) -- (G) -- (H) -- (A) -- (B) -- (C);
  \draw[draw,fill=catmid] (C) -- (E) -- (F) -- (C);
}
\newcommand{\octosliceD}{%
  \draw[draw,fill=catleft](D) -- (E) -- (D);
  \draw[draw, fill=catright](F) -- (G) -- (H) -- (A) -- (B) -- (C) -- (D);
  \draw[draw,fill=catmid] (D) -- (E) -- (F) -- (D);
}

\newcommand{\octosliceG}{%
  \draw[draw,fill=catleft](A) -- (B) -- (C) -- (D) -- (E) -- (G) -- (F) -- (G) -- (H) -- (A);
  \draw[draw,fill=catmid] (G) -- (E) -- (F) -- (G);
}

\newcommand{\octosliceH}{%
  \draw[draw,fill=catleft](H) -- (A) -- (B) -- (C) -- (D) -- (E) --(H) ;
  \draw[draw, fill=catright](F) -- (G) -- (H) -- (A)-- (H);
  \draw[draw,fill=catmid] (H) -- (E) -- (F) -- (H);
}
\newcommand{\octoslice}[1]{%
  \draw[draw,fill=catright](A) -- (B) -- (C) -- (D) -- (E) -- ({#1}) -- (F) -- (G) -- (H) -- (A);
}
\newcommand{\octooslice}[1]{%
  \draw[draw,fill=catright](1) -- (2) -- (3) -- (4) -- (5) -- ({#1}) -- (6) -- (7) -- (8) -- (1);
}


\newcommand{\dycksplit}[2]{%
    \textbf{\textcolor{catmid}{(}} \textcolor{catleftdark}{\{\D_{#1}\}} \textbf{\textcolor{catmid}{)}} \textcolor{catright}{\{\D_{#2}\}}
    }


\newcommand{\pslice}[1]{%
    \penta
    \draw \foreach \x/\y in {#1} {(\x)--(\y)};

}
\newcommand{\cool}[1]{\operatorname{cool}(#1)}
\newcommand{\complementbit}[1]{\operatorname{complement}(#1)}
\newcommand{\lshiftindex}[3][]{\operatorname{left}_{#1}(#2,#3)}
\newcommand{\leftshift}[2][]{\operatorname{left}_{#1}(#2)}
\newcommand{\rrot}[2][]{\operatorname{rrot}_{#1}(#2)}
\newcommand{\Lukas}[1]{\mathcal{L}(#1)}
\DeclareMathOperator{\scut}{scut}
\DeclareMathOperator{\tail}{tail}
\DeclareMathOperator{\nextTreez}{nextTree}
\DeclareMathOperator{\coolCatz}{coolCat}


\def\pcolor{\node[style={fill=lightpurple}]{P};}
\title{Cooler than Cool: \\ Cool-Lex Order for Generating New Combinatorial Objects}
\author{Paul Lapey}
