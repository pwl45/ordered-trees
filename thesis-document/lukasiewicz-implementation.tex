This chapter gives a loopless implementation of the successor rule in \ref{eq:luka} as well as a simplified implementation of the algorithm for the special case of Motzkin words. Section \ref{sec:luka_ll} gives a loopless implementation of \eqref{eq:luka} using an integer linked list representation of a Łukasiewicz word. Section \ref{sec:coolMotz} gives a loopless implemmentation of \eqref{eq:luka} for the special case of Motzkin words.  The algorithm for Motzkin words uses an integer array represeentation of Motzkin words and evaluates at most 3 conditionals per generated string. 


\section{Generating Łukasiewicz Words in Linked Lists}\label{sec:luka_ll}
Implementing left-shifts in an array-based representation of Łukasiewicz words with unrestricted content would almost certainly require some form of loop to implement left-shifts.  In particular, consider the case of $\lshiftindex[\alpha]{1}{k}$ where $\alpha=1,2,3,4,5,\cdots,k$.  This would require shifting each of the first $k$ symbols down one index and therefore require $O(k)$ time.  This is not a constant time operation if $k$ is not a constant.  In particular, a case like this would necessitate worst-case linear time to perform a single iteration of \eqref{eq:luka}.
However, using a linked-list of integers to represent Łukasiewicz words, the left-shifts in \eqref{eq:luka} can be implemented looplessly.  If pointers to the nodes at positions $i$ and $j$ are present, then $\lshiftindex{i}{j}$ simply removes the node at position $i$ and re-inserts it before the node at position $j$.  Thus, a linked-list representation of Łukasiewicz words may allow for a loopless implementation of the successor rule in $\ref{eq:luka}$.

\subsection{Implementation Observations}
The algorithm in \ref{fig:lukaCode} takes advantage of the following observations about equation \ref{eq:luka}:

\begin{enumerate}
    \item In all cases, $\luka{\alpha}$ shifts a single symbol at most 2 symbols past the first increase in $\alpha$ to either the first or second position in $\alpha$.  
    \item Given $m$, a pointer to $\alpha_m$ and a pre-calculated value of $\sum{\rho}$, the correct shift to perform can be determined and executed looplessly.
    \item Case \ref{eq:prefix_n_2} occurs only when $\alpha$ is in descending order and is guaranteed to create an increase at position 2.

    \item Shifting a symbol from position $m+2$ preserves the increase at position $m+1$.  The increase at position $m+1$ remains the first increase in $\alpha$ unless the shift creates an increase at the front of the string.

    \item Shifting a symbol from position $m+1$ creates an increase at position $m+2$ if $\alpha_m < \alpha_{m+2}$.  This becomes the new first increase in the string unless the shift creates an increase at the front of the string.

    \item In the case where a symbol is shifted from position $m+1$, $\alpha_m >= \alpha_{m+2}$, and the shift does not create an increase at the front of the string, the new first increase is whatever the previous second increase in the string was previously. \

Shifts from position $m+1$ occurr either when 
        \begin{enumerate}
            \item $\alpha_m < \alpha_{m+2}$: the new first increase is at position $m+2$.
            \item $m=n-1$: the new first increase is either at the front of the string or does not exist
	    \item $\alpha_m >= \alpha_{m+2}$ and $\alpha_{m+2} = 0$: the new first increase is either at the front of the string or at the previous second increase.  Since $\alpha_{m+2}=0$, if no increase is created at the front of the string, all symbols between $\alpha_{m+2}$ and the new first increase must be zero \label{reasonforstack}
        \end{enumerate}
    \item If shifting creates an increase at the front of the string, the new $\sum{\rho}$ is equal to $\alpha_1$
    \item If shifting does not create an increase at the front of the string, the new $\sum{\rho}$ is equal to its prior value plus the value of the symbol that was shifted.
         
\end{enumerate}

The observation in \ref{reasonforstack} necessitates keeping track of all increases in $\alpha$ in order to guarantee the ability to determine the new first increase in constant time (i.e., without scanning the string).  This is possible in constant time since left-shifting a symbol from position $m,m+1,$ or $m+2$ will never affect any increases past index $m+3$.  Thus, a stack-like data structure containing pointers to ``increase'' nodes and the indices at which they occur is maintained throughout the algorithm's execution.  This requires order n additional space.

% -----------------------------work in progress---------------------------------
    
\begin{figure}[H]
    \begin{subfigure}[]{.5\textwidth}
    \begin{center}
        \begin{Verbatim}








typedef struct ll_node {
    int data;
    struct ll_node* prev;
    struct ll_node* next;
} ll_node;

typedef struct inc {
    struct ll_node* node;
    int index;
} inc;
        \end{Verbatim}
            
    \end{center}

    \caption{struct definitions for linked list Łukasiewicz word representation and the increase stack.}
    \label{fig:lukaStruct}
    \end{subfigure}
    \begin{subfigure}[]{.5\textwidth}
    \begin{center}
        \begin{Verbatim}
void lshift(ll_node* insert, ll_node* shift){
    //remove shift
    ll_node* sprev=shift->prev;
    ll_node* snext=shift->next;
    sprev->next=snext;
    if(snext)
	snext->prev=sprev;

    //insert shift before insert
    ll_node* iprev=insert->prev;
    shift->prev=iprev;
    if(iprev)
	iprev->next=shift;

    shift->next=insert;
    insert->prev=shift;
}
        \end{Verbatim}
    \end{center}

\cprotect\caption{Function for left-shifting a linked list node \verb$shift$ to before \verb$insert$}
    \label{fig:lukaHelpers}
    \end{subfigure}

    \caption{Functions and structs used for the algorithm in \ref{fig:lukaCode}}

\end{figure}

Figure \ref{fig:lukaCode} gives a loopless C implementation of the successor rule in \eqref{eq:luka}.  The while loop iterates until the Łukasiewicz word contains no increases and therefore is in descending order, which occurs at the end of a non-cyclic ordering.  The \verb$inc* incs$ is used as a stack which stores pairs of linked list nodes and indices.  The \verb$ll_node* $ is the linked list node corresponding to the first increase in $\alpha$.  

\begin{figure}
\begin{Verbatim}
void luka_ll(ll_node* hd, ll_node* tl, int n, void (*visit)(ll_node* hd, int n)){
    inc* incs = (inc*) calloc(n, sizeof(inc)); //stack of (node, index) pairs
    int nincs=1; //number of increases

    ll_node *shift_node, *insert_node, *x, *xn;
    int prefix_sum,m,insert_index;

    incs[0] = (inc) {.node=tl,.index=n-1}; //cool struct initializer syntax

    while(nincs){
	m = incs[nincs-1].index;
	x=incs[nincs-1].node; //node at index m+1
	xn=x->next; //node at index m+2

	if(m >= n-1 || xn->data > x->prev->data || (prefix_sum == m && xn->data == 0)){
	    if(m >= n-1 || xn->data > x->data || xn->data == 0){ //increase removed
		nincs--;
	    }else{ //increase kept
		incs[nincs-1].node=x->next;
		incs[nincs-1].index++;
	    }
	    shift_node=x;
	}
	else{ 
	    shift_node=xn; //shift x+1...
	    incs[nincs-1].index++;
	    if(xn->next && xn->next->data > xn->data && xn->next->data <= x->data){
		incs[nincs-2] = incs[nincs-1];
		nincs--;
	    }
	}

	insert_index=!(shift_node->data); //bang
	if(insert_index){
	    insert_node=hd->next;
	}else{
	    insert_node=hd;
	    hd=shift_node;
	}

	lshift(insert_node,shift_node);
	if(insert_index != m && (shift_node->data < insert_node->data)){
	    prefix_sum=hd->data;
	    incs[nincs++]= (inc) {.node = insert_node, .index=insert_index+1};
	}else{
	    prefix_sum+=shift_node->data;
	}

	visit(hd,n);
    }
}
\end{Verbatim}

\caption{Function for looplessly generating all Łukasiewicz words with a fixed set of content.}
\label{fig:lukaCode}
\end{figure}

First, the algorithm pops an increase off of the \verb$incs$ stack, storing the increase's index in \verb$m$ and the increase's node in \verb$x$.  Next, the algorithm determines which node is to be shifted, denoted via \verb$shift_node$.  This is accomplished in the main if-else block.  The first if statement evaluates the condition in \eqref{eq:prefix_m1_1} and if it is true sets \verb$shift_node$ to \verb$x$,the node at index $m+1$. Otherwise, it sets \verb$shift_node$ to \verb$xn$, the node at index $m+2$.  In both cases, the algorithm checks the values of the nodes adjacent to the \verb$shift_node$ to determine if an increase has been modified or removed by shifting \verb$shift_node$.

The next block of code uses the fact that the index to shift to is $1$ if a $1$ is being shifted and $0$ otherwise.  The head of the list, \verb$hd$, is updated if a node is shifted to index 0.  Finally, the algorithm executes the left-shift with \verb$insert_node$ and \verb$shift_node$, checks if an increase was created at the front of the list, and updates the \verb$prefix_sum$.  This algorithm is clearly loopless, as \verb$lshift$ is a constant time operation and the function \verb$luka_ll$ has no inner loops.

\section{Generating Motzkin Words in Arrays}\label{sec:coolMotz}
Since Łukasiewicz words are a generalization of Motzkin words, the same algorithm can be used to generate Motzkin words by restricting the content set S to be strictly zeroes, ones, and twos.  However, the additional restrictions on Motzkin words allow for a simpler implementation of the rule.  In particular, the content restriction obviates the need for an increase stack.   Pseudocode for loopless generation of Motzkin words is given below in Fig. \ref{fig:motzkinAlg}. 


% \iffalse
\begin{figure}[H]
    \centering
        \begin{algorithm}[H]
        \begin{algorithmic}
        \Function{coolMotzkin}{$s,t$}
        \EndFunction{}
         
        \State $n \gets 2*s+t$
        \State $b \gets 2^1 0^1 2^{s-1} 1^t 0^{s-1}$
        \State $x \gets 3$
        \State $y \gets 2$
        \State $z \gets 2$
        
        \State \visit{$b$}
        
        \While{$x <= n$}
            \State $q \gets b_{x-1}$
            \State $r \gets b_x$
            \vspace{.4em} 
            \State $b_x\gets b_{x-1}$
            \State $b_y\gets b_{y-1}$
            \State $b_z\gets b_{z-1}$
            \State $b_1\gets r$
            \vspace{.4em} 
            \State $x \gets x+1$
            \State $y \gets y+1$
            \State $z \gets y+1$
            
            \vspace{.4em} 
            \If{$b_x = 0$}
                \If{$z-2 > x-y$}
                    \State $b_1=2$
                    \State $b_2=0$
                    \State $b_x=r$
                    \State $x \gets 3$
                    \State $y \gets 2$
                    \State $z \gets 2$
                \Else 
                    \State $x \gets x+1$
                
                \EndIf
                \ElsIf{$q \ge b[x]$}
                    \State $b_x \gets 2$     
                    \State $b_{x-1} \gets 1$     
                    \State $b_1 \gets 1$     
                    \State $z \gets 1$     
                
            \EndIf
            \visit{$b$}
        \EndWhile
        \end{algorithmic}
        % I don't know how to get rid of this "a" here
        \caption{Motzkin}
        \end{algorithm}
    \caption{Pseudocode algorithm for loopless enumeration of Motzkin words}
    \label{fig:motzkinAlg}
\end{figure}
% \fi
