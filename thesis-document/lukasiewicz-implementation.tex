This chapter gives a loopless implementation of the successor rule in \ref{eq:luka} as well as a simplified implementation of the algorithm for the special case of Motzkin words which evaluates at most 3 conditionals per generated string. 


\section{Generating Lukasiewicz Words in Linked Lists}
Implementing left-shifts in an array-based representation of Lukasiewicz words would almost certainly require some form of loop left-shifts.  In particular, left-shifting a symbol from an position $i$ in the array to front of the array would require shifting each of the first $i$ symbols down.  This would necessitate worst-case linear time to perform a single iteration. 
However, using a linked-list of integers to represent Lukasiewicz words, left-shifts can be implemented looplessly as performing a left-shift requires only removing the node to be shifted from the list it and re-inserting it at position 1 or 2.  Thus, a linked-list representation of Lukasiewicz words allows for a loopless implementation of the successor rule in $\ref{eq:luka}$.

\subsection{Observations}
The algorithm in \ref{fig:lukaAlg} takes advantage of the following observations about equation \ref{eq:luka}:

\begin{enumerate}
    \item In all cases, $\luka{\alpha}$ shifts a single symbol at most 2 symbols past the first increase in $\alpha$ to either the first or second position in $\alpha$.  
    \item Given $m$, a pointer to $\alpha_m$ and a pre-calculated value of $\sum{\rho}$, the correct shift to perform can be determined and executed looplessly.
    \item Case \ref{eq:prefix_n_2} occurs only when $\alpha$ is in descending order and is guaranteed to create an increase at position 2.

    \item Shifting a symbol from position $m+2$ preserves the increase at position $m+1$.  The increase at position $m+1$ remains the first increase in $\alpha$ unless the shift creates an increase at the front of the string.

    \item Shifting a symbol from position $m+1$ creates an increase at position $m+2$ if $\alpha_m < \alpha_{m+2}$.  This becomes the new first increase in the string unless the shift creates an increase at the front of the string.

    \item In the case where a symbol is shifted from position $m+1$, $\alpha_m >= \alpha_{m+2}$, and the shift does not create an increase at the front of the string, the new first increase is whatever the previous second increase in the string was previously.

Shifts from position $m+1$ occurr either when 
        \begin{enumerate}
            \item $\alpha_m < \alpha_{m+2}$: the new first increase is at position $m+2$.
            \item $m=n-1$: the new first increase is either at the front of the string or does not exist
            \item $\alpha_m >= \alpha_{m+2}$ and $\alpha_{m+2} = 0$: the new first increase is either at the front of the string or at the previous second increase.  Since $\alpha_{m+2}=0$, if no increase is created at the front of the string, all symbols between $\alpha_{m+2}$ and the new first increase must be zero
        \end{enumerate}
    \item If shifting creates an increase at the front of the string, the new $\sum{\rho}$ is equal to $\alpha_1$
    \item If shifting does not create an increase at the front of the string, the new $\sum{\rho}$ is equal to its prior value plus the value of the symbol that was shifted.
         
\end{enumerate}

This final observation necessitates keeping track of all increases in $\alpha$ in order to guarantee the ability to determine the new first increase in constant time (i.e., without scanning the string).  This is possible in constant time since left-shifting a symbol from position $m,m+1,$ or $m+2$ will never affect any increases past index $m+3$.  Thus, a stack-like data structure containing pointers to ``increase'' nodes and the indices at which they occur is maintained throughout the algorithm's execution.  This requires order n additional space.

-----------------------------work in progress---------------------------------
    
\begin{figure}[H]
    \begin{subfigure}[]{.5\textwidth}
    \begin{center}
        \begin{Verbatim}
typedef struct ll_node {
    int data;
    struct ll_node* prev;
    struct ll_node* next;
} ll_node;

typedef struct inc {
    struct ll_node* node;
    int index;
} inc;
        \end{Verbatim}
            
    \end{center}

    \caption{struct definitions for linked list Lukasiewicz word representation and the increase stack.}
    \label{fig:lukaStruct}
    \end{subfigure}
    \begin{subfigure}[]{.5\textwidth}
    \begin{center}
        \begin{Verbatim}
void lshift_ll(ll_node* insert_node, ll_node* shift_node){
    //remove shift_node
    ll_node* sprev=shift_node->prev;
    ll_node* snext=shift_node->next;
    if(sprev){
	sprev->next=snext;
    }
    if(snext){
	snext->prev=sprev;
    }

    //insert shift_node before insert_node
    ll_node* iprev=insert_node->prev;
    shift_node->prev=iprev;
    if(iprev){
	iprev->next=shift_node;
    }

    shift_node->next=insert_node;
    insert_node->prev=shift_node;
}

        \end{Verbatim}
            
    \end{center}

\caption{helper function for left-shifting a linked list node}
    \label{fig:lukaHelpers}
    \end{subfigure}

\end{figure}


% \begin{figure}[H]
% \begin{Verbatim}
% nop
% \end{Verbatim}

% \label{fig:lukaHelpers}
% \end{figure}


\begin{figure}[H]
\begin{Verbatim}
void luka_ll(ll_node* hd, ll_node* tl, int n, void (*visit)(ll_node* hd)){
    inc* incs = (inc*) calloc(n/2, sizeof(inc)); //stack of (node, index) pairs
    int nincs=1; //number of increases
    incs[0] = (inc) {.node=tl,.index=n-1}; //cool struct initializer syntax

    ll_node *shift_node, *insert_node;
    int prefix_sum,prefix_len,insert_index;

    while(nincs){
	ll_node* m=incs[nincs-1].node;
	prefix_len = incs[nincs-1].index;
	if(prefix_len >= n-1){ // increase removed
	    nincs--;
	    shift_node=m;
	}
	else if(m->next->data > m->prev->data){
	    if(m->next->data > m->data){ //increase removed
		nincs--;
	    }else{ //increase kept
		incs[nincs-1].node=m->next;
		incs[nincs-1].index++;
	    }
	    shift_node=m;
	}
	else{
	    if(prefix_sum > prefix_len || m->next->data > 0){ //not tight
		shift_node=m->next; //shift m+2...
		incs[nincs-1].index++; //same increase, different location (shifted down by one)
		if(prefix_len < n-2 && m->next->next->data > m->next->data && 
                 m->next->next->data <= m->data){ //hideous line of code
		    incs[nincs-2] = incs[nincs-1];
		    nincs--;
		}
	    }else{ //tight; increase removed
		nincs--;	
		shift_node=m;
	    }
	}

	insert_index=!(shift_node->data); //bang
	if(insert_index){
	    insert_node=hd->next;
	}else{
	    insert_node=hd;
	    hd=shift_node;
	}
	lshift_ll(insert_node,shift_node);
	if(insert_index != prefix_len && (shift_node->data < insert_node->data)){
	    prefix_sum=hd->data;
	    incs[nincs++]= (inc) {.node = insert_node, .index=insert_index+1};
	}else{
	    prefix_sum+=shift_node->data;
	}
	visit(hd);
    }
}
\end{Verbatim}

\caption{work in progress. At least it already fits on one page.}
\label{fig:lukaAlg}
\end{figure}


\section{Generating Motzkin Words in Arrays}
Since Lukasiewicz words are a generalization of Motzkin words, the same algorithm can be used to generate Motzkin words by restricting the content set S to be strictly zeroes, ones, and twos.  However, the additional restrictions on Motzkin words allow for a simpler implementation of the rule.   Pseudocode for loopless generation of Motzkin words is given below in Fig. \ref{fig:motzkinAlg}. 


% \iffalse
\begin{figure}[H]
    \centering
        \begin{algorithm}[H]
        \begin{algorithmic}
        \Function{coolMotzkin}{$s,t$}
        \EndFunction{}
         
        \State $n \gets 2*s+t$
        \State $b \gets 2^1 0^1 2^{s-1} 1^t 0^{s-1}$
        \State $x \gets 3$
        \State $y \gets 2$
        \State $z \gets 2$
        
        \State \visit{$b$}
        
        \While{$x <= n$}
            \State $q \gets b_{x-1}$
            \State $r \gets b_x$
            \vspace{.4em} 
            \State $b_x\gets b_{x-1}$
            \State $b_y\gets b_{y-1}$
            \State $b_z\gets b_{z-1}$
            \State $b_1\gets r$
            \vspace{.4em} 
            \State $x \gets x+1$
            \State $y \gets y+1$
            \State $z \gets y+1$
            
            \vspace{.4em} 
            \If{$b_x = 0$}
                \If{$z-2 > x-y$}
                    \State $b_1=2$
                    \State $b_2=0$
                    \State $b_x=r$
                    \State $x \gets 3$
                    \State $y \gets 2$
                    \State $z \gets 2$
                \Else 
                    \State $x \gets x+1$
                
                \EndIf
                \ElsIf{$q \ge b[x]$}
                    \State $b_x \gets 2$     
                    \State $b_{x-1} \gets 1$     
                    \State $b_1 \gets 1$     
                    \State $z \gets 1$     
                
            \EndIf
            \visit{$b$}
        \EndWhile
        \end{algorithmic}
        % I don't know how to get rid of this "a" here
        \caption{Motzkin}
        \end{algorithm}
    \caption{Pseudocode algorithm for loopless enumeration of Motzkin words}
    \label{fig:motzkinAlg}
\end{figure}
% \fi
